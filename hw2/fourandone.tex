\documentclass[11pt]{article}
        \usepackage[T1]{fontenc} % Font styles
	\usepackage[utf8]{inputenc} % Special characters "ä, ö, ü, ß" made possible
        \usepackage[left=2cm,right=2cm,top=2cm,bottom=2cm]{geometry} % page setup
        \usepackage[doublespacing]{setspace} % set line spacing
        \usepackage{multicol} % multiple columns: \begin{multicols}{number}....\end{multicols}
        \usepackage{csquotes} % \blockquote{} command for long quotes
	\usepackage[german]{babel} % Language (deutsch: ngerman)
	\usepackage{amsmath} % Add formulas to document
	\usepackage{graphicx} % Add graphics to document
        \usepackage{cancel} % Cancel out terms in equations
        \usepackage{xcolor} % Color output [see commands]
        \usepackage{array} % keep equations aligned by inserting & at the alignment point
        \usepackage{unicode-math}
        \usepackage{longtable} % tables longer than one page
        \usepackage{booktabs} % fancy tables from r
        \usepackage{dcolumn} % also r tables
        \usepackage{rotating} % rotated tables
        \usepackage[toc,page]{appendix} % appendix
        \usepackage{fancyhdr} % header and footer
            \pagestyle{fancy}
            \rhead{Chapter \leftmark}
            \renewcommand\headrulewidth{1pt}
            \setlength{\headheight}{14pt}
	\usepackage[style=authoryear, backend=bibtex]{biblatex} % citation that works 
            \bibliography{bibliography} % declare bibliography file name without .bib extention
% Declaration of commands
            \newcommand{\lagr}{nn\mathcal{L}} % \lagr for lagrangian [I need this all the time so makes sense to declare shortcut]
            \newcommand\Ccancel[2][black]{\renewcommand\CancelColor{\color{#1}}\cancel{#2}} % cancel with colors \Ccancel[color]{...}; default: black

\pagenumbering{arabic}
\setlength{\parskip}{0pt}

\begin{document}


\section{Question 1}
Let $X = (X_1, X_2, ..., X_K)'$ be a random vector with mean zero and covariance matrix
\begin{equation}
  \label{eq:1}
  \Gamma = \mathbf{E} XX'
\end{equation}

Assume $\Gamma$ is singular. Then there exists an eigenvalue $\lambda_1 = 0$ of $\Gamma$ with corresponding eigenvector $v_1 = (v_{11}, v_{12}, ..., v_{1K})'$. We know that
\begin{equation}
  \label{eq:2}
  \Gamma v_1 = \lambda_1 v_1 = \mathbf{0_K}
\end{equation}
  
where $\mathbf{0_K}$ denotes the zero vector of length $K$. Henceforth it will be denoted by $\mathbf{0}$. It is equivalent to say that
\begin{equation}
  \label{eq:3}
  \mathbf{E} XX'v_1 = \mathbf{0}.
\end{equation}

We can multiply $v_1'$ from the left to get the variance of $v_1'X$.
\begin{equation}
\begin{split}
  \label{eq:4}
  \mathbf{E} v_1'XX'v_1 &= \mathbf{E} (v_1'X)(v_1'X)' = Var(v_1'X)\\
  &= v_1'\mathbf{0} = 0
\end{split}
\end{equation}

Since the variance is zero, we conclude that $v_1'X$ is deterministic and thus equal to a constant $d$. Thus we find
\begin{equation}
  \label{eq:5}
  v_1'X = v_{11} X_1 + v_{12} X_2 + \dots + v_{1K} X_K = d
\end{equation}

We can rearrange (\ref{eq:5}) to

\begin{equation}
  \label{eq:6}
  d - v_{1j} X_j = v_{11} X_1 + \dots + v_{1j-1} X_{j-1} + v_{1j+1}X_{j+1}+ \dots + v_{1K} X_K, \ j \in (2, K-1)
\end{equation}
Without loss of generality $j$ can be equal to $1$ or $K$ as well by deducting the appropriate $v$ and $X$ instead when moving from~(\ref{eq:5}) to (\ref{eq:6}). Now divide by $v_{1j}$ and define $\alpha_i := - \frac{v_{1i}}{v_{1j}}$ for $i = 1, ..., K$. It follows that $\alpha_j = -1$. Thus
\begin{equation}
  \label{eq:7}
 X_j + \frac{d}{v_{1j}}  = \alpha_1 X_1 + \alpha_2 X_2 + \dots + \alpha_K X_K
\end{equation}

Now let $c := -\frac{d}{v_{1j}}$. Then
\begin{equation}
  \label{eq:8}
  X_j = c + \alpha_1 X_1 + \alpha_2 X_2 + \dots + \alpha_{j-1} X_{j-1} + \alpha_{j+1} X_{j+1}+ \dots + \alpha_K X_K
\end{equation}


$
\printbibliography
\end{document} % Nothing gets printed after here 
%%% Local Variables:
%%% mode: latex
%%% TeX-master: t
%%% End:

%%% Local Variables:
%%% mode: latex
%%% TeX-master: t
%%% End:
