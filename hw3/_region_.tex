\message{ !name(three_four.tex)}\documentclass[11pt]{article}
        \usepackage[T1]{fontenc} % Font styles
	\usepackage[utf8]{inputenc} % Special characters "ä, ö, ü, ß" made possible
        \usepackage[left=2cm,right=2cm,top=2cm,bottom=2cm]{geometry} % page setup
        \usepackage[doublespacing]{setspace} % set line spacing
        \usepackage{multicol} % multiple columns: \begin{multicols}{number}....\end{multicols}
        \usepackage{csquotes} % \blockquote{} command for long quotes
	\usepackage[german]{babel} % Language (deutsch: ngerman)
	\usepackage{amsmath} % Add formulas to document
	\usepackage{graphicx} % Add graphics to document
        \usepackage{cancel} % Cancel out terms in equations
        \usepackage{amssymb}
\usepackage{xcolor} % Color output [see commands]
        \usepackage{array} % keep equations aligned by inserting & at the alignment point
        \usepackage{longtable} % tables longer than one page
        \usepackage{booktabs} % fancy tables from r
        \usepackage{dcolumn} % also r tables
        \usepackage{rotating} % rotated tables
        \usepackage[toc,page]{appendix} % appendix
        \usepackage{fancyhdr} % header and footer
            \pagestyle{fancy}
            \rhead{Chapter \leftmark}
            \renewcommand\headrulewidth{1pt}
            \setlength{\headheight}{14pt}
	\usepackage[style=authoryear, backend=bibtex]{biblatex} % citation that works 
            \bibliography{bibliography} % declare bibliography file name without .bib extention
% Declaration of commands
            \newcommand{\lagr}{nn\mathcal{L}} % \lagr for lagrangian [I need this all the time so makes sense to declare shortcut]
            \newcommand\Ccancel[2][black]{\renewcommand\CancelColor{\color{#1}}\cancel{#2}} % cancel with colors \Ccancel[color]{...}; default: black

\pagenumbering{arabic}
\setlength{\parskip}{0pt}

\begin{document}

\message{ !name(three_four.tex) !offset(-3) }

\thispagestyle{empty}
\begin{center}
  \section*{3/4} % there is some stuff with titlepage setup around but I find this to be more convenient as you can do whatever you want here. Add date/time, Authornames, space...
  \textbf{}\\
  \large{Daniel Winkler}
  
  {\large \today\par}
\end{center}

\section{Construction of two random variables with given covariance}
\label{sec:constr-two-rand}



Let $Z_{1,t},\ Z_{2,t}$ be i.i.d. $N(0,1)$. We want to construct two variables $X_t,\ Y_t$ s.t.
\begin{equation}
  \label{eq:1}
  \begin{bmatrix}
    X\\
    Y
  \end{bmatrix} \sim
  N \left(
    \begin{bmatrix}
      \mu_x\\
      \mu_y
    \end{bmatrix},
    \begin{bmatrix}
      \sigma^2_x & \sigma_{x,y}\\
      \sigma_{x,y} & \sigma_y^2
    \end{bmatrix}
\right)
\end{equation}

Let $X \sim N(\mu_x, \sigma^2_x)$. Then:
\begin{equation}
  \label{eq:2}
  \frac{X-\mu_x}{\sigma_x} \sim N(0,1) \equiv Z_{1,t}
\end{equation}

It follows that:
\begin{equation}
  \label{eq:3}
  \mu_x + \sigma_x Z_{1,t} \sim N(\mu_x, \sigma^2_x)
\end{equation}

Now let $Z_{3,t} = \rho Z_{1,t} + \sqrt{1-\rho^2} Z_{2,t}$. We can show that this is also a $N(0,1)$:
\begin{equation}
  \label{eq:4}
  \begin{split}
    &\rho Z_{1,t} + \sqrt{1-\rho^2} Z_{2,t}\\
    &=\rho N(0,1) + \sqrt{1-\rho^2} N(0,1)\\
    &=N(0,\rho^2) + N(0, 1-\rho^2)
  \end{split}
\end{equation}
As $Z_{1,t}$ and $Z_{2,t}$ are independent:
\begin{equation}
  \label{eq:5}
    N(0,\rho^2) + N(0, 1-\rho^2) = N(0,1-\rho^2+\rho^2) = N(0,1)
\end{equation}

Thus:
\begin{equation}
  \label{eq:6}
  \mu_y + \sigma_y Z_{3,t} \sim N(\mu_y, \sigma^2_y)
\end{equation}

We still need to show that $cov(X,Y) = \sigma_{x,y} = \sigma_x \sigma_y \rho$.
\begin{equation}
  \label{eq:7}
  \begin{split}
    cov(X,Y) &= \mathbb{E}[(X - \mathbb{E}[X])(Y - \mathbb{E}[Y])]\\
    &= \mathbb{E}[(X - \mu_x)(Y-\mu_y)]\\
    &= \mathbb{E}[(\sigma_x Z_{1,t} + \mu_x - \mu_x)(\sigma_y (\rho Z_{1,t} + \sqrt{1-\rho^2} Z_{2,t}) + \mu_y - \mu_y)]\\
    &= \sigma_x \sigma_y \mathbb{E}[\rho Z^2_{1,t} + \sqrt{1-\rho^2}Z_{1,t}Z_{2,t}]
  \end{split}
\end{equation}
Lets look at the second part of the sum. $\sqrt{1-\rho^2}$ is a constant and can thus be taken out of the expectation. As $Z_{1,t}$ and $Z_{2,t}$ are independent:
\begin{equation}
  \label{eq:8}
  \begin{split}
    \mathbb{E}[Z_{1,t}Z_{2,t}] = \mathbb{E}[Z_{1,t}] \mathbb{E}[Z_{2,t}] = 0 
  \end{split}
\end{equation}
where the last equality follows from the definition of the variables.
Thus,
\begin{equation}
  \label{eq:9}
  \begin{split}
  &\sigma_x \sigma_y \mathbb{E}[\rho Z^2_{1,t} + \sqrt{1-\rho^2}Z_{1,t}Z_{2,t}]\\
  &= \sigma_x \sigma_y \rho \mathbb{E}[Z^2_{1,t}]
\end{split}
\end{equation}
As $Z^2_{1,t}$ is the square of one normally distributed variable, we know that $Z^2_{1,t} \sim \chi^2_1$. Since the expected value of a $\chi^2$ distributed variable is just its degrees of freedom $\mathbb{E}[Z^2_{1,t}] = 1$. Thus:
\begin{equation}
  \label{eq:10}
  cov(X,Y) = \sigma_x \sigma_y \rho
\end{equation}

\section{Conditional Expectation and Variance}
\label{sec:cond-expect-vari}

\subsection{Conditional Expectation of Y|X}
\label{sec:cond-expect}
\begin{equation}
  \label{eq:11}
  \begin{split}
    \mathbb{E}[Y|X] &= \mathbb{E}[\mu_y + \sigma_y (\rho Z_{1,t} + \sqrt{1-\rho^2} Z_{2,t})|X]\\
    &= \mathbb{E}[\mu_y + \sigma_y (\rho \frac{X-\mu_x}{\sigma_x} + \sqrt{1-\rho^2} Z_{2,t})|X]\\
  \end{split}
\end{equation}
$\mathbb{E}[\frac{X-\mu_x}{\sigma_x}|X] = \frac{X-\mu_x}{\sigma_x}$. Note that in this case the realization of $X$ is given, thus the fraction is a constant. $\mu_y,\ \sigma_y \text{ and } \rho$ are chosen constants and thus independent of $X$. Thus we get:
\begin{equation}
  \label{eq:12}
  \begin{split}
    &\mathbb{E}[\mu_y + \sigma_y (\rho \frac{X-\mu_x}{\sigma_x} + \sqrt{1-\rho^2} Z_{2,t})|X]\\
    & = \mu_y + \sigma_y (\rho \frac{X-\mu_x}{\sigma_x} + \sqrt{1-\rho^2} \mathbb{E} [Z_{2,t}|X])
  \end{split}
\end{equation}
Since $Z_{1,t},\ Z_{2,t}$ are independent and the parameters $\mu_x,\ \sigma_x$ are chosen independently of $Z_{2,t}$:
\begin{equation}
  \label{eq:13}
  \mathbb{E}[Z_{2,t}|X]= \mathbb{E}[Z_{2,t}] = 0
\end{equation}
Thus,
\begin{equation}
  \label{eq:15}
  \mathbb{E}[Y|X] = \mu_y + \sigma_y (\rho \frac{X-\mu_x}{\sigma_x}) = \mu_y + \sigma_y \sigma_x \rho \frac{X-\mu_x}{\sigma_x^2} = \mu_y + \sigma_{xy}\frac{X-\mu_x}{\sigma_x^2}
\end{equation}

\subsection{Conditional Variance of Y|X}
\label{sec:cond-vari-yx}
\begin{equation}
  \label{eq:16}
    Var(Y|X) = Var(\mu_y + \sigma_y (\rho \frac{X-\mu_x}{\sigma_x} + \sqrt{1-\rho^2} Z_{2,t})|X)
\end{equation}
As $\mu_y,\ \sigma_y,\ \rho$ and $\frac{X-\mu_x}{\sigma_x}|X$ are constants and thus have no variance:
\begin{equation}
  \label{eq:17}
  = Var(\sigma_y \sqrt{1-\rho^2} Z_{2,t}|X) = Var(\sigma_y \sqrt{1-\rho^2} Z_{2,t})
\end{equation}
because of independence of $X$ and the constants as well as $Z_{2,t}$.
\begin{equation}
  \label{eq:18}
  Var(\sigma_y \sqrt{1-\rho^2} Z_{2,t}) = \sigma_y^2 (1-\rho^2) Var(Z_{2,t}) = \sigma_y^2 - \sigma_y^2 \rho^2
\end{equation}
Multiplying and dividing the second part by $\sigma_x^2$ yields:
\begin{equation}
  \label{eq:19}
  Var(Y|X) = \sigma_y^2 - \frac{\sigma_x^2 \sigma_y^2 \rho^2}{\sigma_x^2} = \sigma_y^2 - \frac{\sigma_{xy}^2}{\sigma_x^2}
\end{equation}



\subsubsection*{Problem 6}
\label{sec:problem-6}

Show that:
\[
  \int_{-\pi}^\pi e^{i(k-h)\lambda} d \lambda = \left\{
    \begin{array}{ll}
      2\pi,\ \text{if} \ k=h,\\
      0,\ \text{otherwise}
    \end{array}
     \right.
\]


For $k\uneq h$e evaluate the indefinite integral:
\begin{equation}
  \label{eq:21}
  \begin{split}
    &\int  e^{i(k-h)\lambda} d \lambda\\
    &= \int cos((k-h) \lambda) + i sin((k-h) \lambda) d \lambda\\
    &= \int cos((k-h) \lambda) d \lambda + i \int sin((k-h)\lambda) d \lambda\\
    &= \frac{sin((k-h) \lambda)}{k-h} - i \frac{cos((k-h) \lambda)}{k-h}
  \end{split}
\end{equation}
Now we evaluate from $-\pi$ to $\pi$:
\begin{equation}
  \label{eq:23}
  \begin{split}
    &\left.  \frac{sin((k-h) \lambda)}{k-h} - i \frac{cos((k-h) \lambda)}{k-h} \right\vert_{-\pi}^\pi\\
    &=  \frac{sin((k-h) \pi)}{k-h} - i \frac{cos((k-h) \pi)}{k-h} - \left( \frac{sin((k-h) (-\pi))}{k-h} - i \frac{cos((k-h) (-\pi))}{k-h}\right)
  \end{split}
\end{equation}



\printbibliography
\end{document} % Nothing gets printed after here 
%%% Local Variables:
%%% mode: latex
%%% TeX-master: t
%%% End:

%%% Local Variables:
%%% mode: latex
%%% TeX-master: t
%%% End:

\message{ !name(three_four.tex) !offset(-238) }
